\chapter{本文重要定理及引理的证明}
\label{cha:proof}

\section{定理~\ref{theo:2:1}~的证明}\label{pr:2:1}
\begin{proof}
我们利用数学归纳法证明。
\par
对$k^\prime = 0$,显然$\|y-u(0)\|_2^2 \leq 1\cdot \|y-u(0)\|_2^2$.
\par
假设对于$k^\prime = 0,\cdots,k$,我们有$\|y-u(k^\prime)\|_2^2 \leq (1-n\lambda_0/2)^{k^\prime} \|y-u(0)\|_2^2$.
\par
对$k^\prime = k+1$,我们令
\[
A_{ir} = \{\exists w:\|w-w_r(0)\|\leq R, \mathbb{I}\{x_i^\top w_r(0)\geq 0\}\neq \mbi \{x_i^\top w\geq 0\}\}
\]
其中$R = \frac{c\lambda_0}{n^2}, c$是一个正常数。我们令$S_i = \{r\in [m]: \mbi\{A_{ir}=0\}\}, S_i^\bot  = [m]\backslash S_i\}$,则以下的引理给出了$S_i^\top$的和的大小的一个界。
\begin{lemma}
至少以概率$1-\delta$有,
\[
\sum_{i=1}^n |S_i^\top|\leq\frac{CmnR}{\delta}, C>0\makebox[3em]{为常数}
\]
\end{lemma}
\par
考虑
\[
\begin{aligned}
u_i(k+1) - u_i(k) & = \frac{1}{\sqrt{m}}\sum_{r=1}^ma_r(\sigma(w_r(k+1)^\top x_i) - \sigma (w_r(k)^\top x_i))\\
 & = \frac{1}{\sqrt{m}}\sum_{r=1}^ma_r(\sigma((w_r(k)-\eta\frac{\partial L(W(k))}{\partial w_r(k)})^\top x_i) - \sigma (w_r(k)^\top x_i))\\
\end{aligned}
\]
\par
我们利用$S_i$将上述等式的右边划分为两部分,
\[
I_1^i \triangleq \frac{1}{\sqrt{m}}\sum_{r\in S_i} a_r(\sigma((w_r(k)-\eta\frac{\partial L(W(k))}{\partial w_r(k)})^\top x_i) - \sigma (w_r(k)^\top x_i))
\]
\[
I_2^i \triangleq \frac{1}{\sqrt{m}}\sum_{r\in S_i^\bot} a_r(\sigma((w_r(k)-\eta\frac{\partial L(W(k))}{\partial w_r(k)})^\top x_i) - \sigma (w_r(k)^\top x_i))
\]
\par
因为ReLU为1-Lipschitz的,并且$|a_r| = 1$,我们有
\[
|I_2^i|\leq \frac{\eta}{\sqrt{m}} \sum_{r\in S_i^\bot }|\bigg(\frac{\partial L(W(k))}{\partial w_r(k)}\bigg)^\top x_i|\leq \frac{\eta|S_i^\bot|}{\sqrt{m}}\max_{r\in [m]}\|\frac{\partial L(W(k))}{\partial w_r(k)}\|_2\leq \frac{\eta|S_i^\bot|\sqrt{n}\|u(k)-y\|_2}{m} 
\]
\par
由于对于$\forall r\in[m]$,$\|w_r(k)-w_r(0)\|\leq R^\prime$,且$R^\prime \le R$,我们有$\mbi\{w_r(k+1)^\top x_i\geq 0\} = \mbi\{w_r(k)^\top x_i \geq 0\}, \forall r \in S_i$,则
\[
\begin{aligned}
I_1^i &= -\frac{\eta}{m}\sum_{j=1}^n x_i^\top x_j (u_j-y_j)\sum_{r\in S_i}\mbi\{w_r(k)^\top x_i\geq 0,w_r(k)^\top x_j \geq 0\}\\
 & -\eta\sum_{j=1}^n (u_j-y_j)(H_{ij}(k) - H_{ij}^\bot(k))
\end{aligned}
\]
其中$H_{ij}(k) = 1/m\sum_{r=1}^m x_i^\top x_j\mbi\{w_r(k)^\top x_i\geq 0, w_r(k)^\top x_j\geq 0\}, H_{ij}^\bot(k) = 1/m\sum_{r\in S_i^\bot} x_i^\top x_j\mbi\{w_r(k)^\top x_i\geq 0, w_r(k)^\top x_j\geq 0\}$。由上面的引理可以知道
\[
\|H*\bot(k)\|_2 \leq \sum_{(i,j)=(1,1)}^{(n,n)}|H_{ij}^\bot (k)\leq \frac{n\sum_{i=1}^n |S_i^\bot|}{m} \leq \frac{Cn^2 mR}{\delta m} \leq \frac{Cn^2R}{\delta}
\]
从而我们有
\[
\|u(k+1)-u(k)\|_2^2 \leq \eta^2\sum_{i=1}^n \frac{1}{m}\bigg(\sum_{r = 1}^m\|\frac{\partial L(W(k))}{\partial w_r(k)}\|_2 \bigg)^2\leq \eta^2 n^2\|y-u(k)\|_2^2
\]
进而通过引入交叉项有
\[
\begin{aligned}
\|y-u(k+1)\|_2^2 & = \|y-u(k) -(u(k+1)-u(k))\|_2^2\\
 & = \|y-u(k)\|_2^2 -2 (y-u(k))^\top (u(k+1)-u(k))+\|u(k+1)-u(k)\|_2^2\\
 & = \|y-u(k)\|_2^2 -2 \eta(y-u(k))^\top H(k)(y-u(k))+2\eta(y-u(k))^\top H(k)^\top (y-u(k)) \\
 & - 2(y-u(k))^\top I_2 + \|u(k+1)- u(k)\|_2^2\\
 & \leq (1-\eta\lambda_0+ \frac{2C\eta n^2R}{\delta} + \frac{2C\eta n^{3/2}R}{\delta}+eta^2 n^2)\|y-u(k)\|_2^2\\
 & \leq (1-\eta\lambda_0/2)\|y-u(k)\|_2^2
\end{aligned}
\]
其中第三个等式成立是对$u(k+1)-u(k)$的分解。
\par
由以上的结果知,对$k^\prime = k+1$命题也成立,所以由归纳法知,定理成立。
\end{proof}


\section{定理~\ref{theo:1}~的证明}\label{pr:1}
\begin{proof}
令$z_i = x_i, \theta = \sum_{l=0}^{r-1}\mathrm{S}_l^s(w)$,其中
\[
	\mathrm{S}_l^s(w) = [\underbrace{0,\cdots,0}_{ls\makebox[1em]{个}0}, w_1,\cdots,w_m,0\cdots,0]\in \mathbb{R}^d
\]
则$F_1(x_i;w) = \langle z_i,\theta \rangle$,
\par
定义$\theta\in \mathbb{R}^d$的经验范数为
\[
\|\theta\|_x^2 = \frac{1}{n}\sum_{i=1}^n |\langle z_i, \theta\rangle|^2
\]
则由于$\hat{\theta} = \sum_{l=0}^{r-1}\mathrm{S}_l^s(\hat{w})$,有
\[
	\frac{1}{n}\sum_{i=1}^n (y_i - \langle z_i, \hat{\theta}\rangle)^2 \leq \frac{1}{n}\sum_{i=1}^n (y_i - \langle z_i, {\theta}\rangle)^2
\]
又$y_i = \langle z_i, {\theta}\rangle + \varepsilon_i$,上式可以整理成为
\begin{equation}\label{eq:18}
\|\hat{\theta}- \theta\|_x^2 \leq \frac{2}{n}\sum_{i=1}^n \varepsilon_i\langle z_i,\hat{\theta}-\theta\rangle
\end{equation}
\par
设$\theta$参数的集合为$\Theta$,定义
\[
	\overline{\Theta_x} := \{\phi = \theta - \theta^\prime: \theta, \theta^\prime \in \Theta, \|\phi\|_x\leq 1\}
\]
\[
	\mathbb{G}_x(\phi) = \frac{1}{n}\sum_{i=1}^n \varepsilon_i\langle z_i,\phi\rangle
\]
则有以下引理成立
\begin{lemma}
对于参数集$\Theta$,我们有
\begin{equation}\label{eq:l4}
\sup_{\hat{\theta}\in \Theta} \sum_{i=1}^n \varepsilon_i\langle z_i,\hat{\theta}-\theta\rangle \leq \|\hat{\theta} - \theta \|_x \sup_{\phi \in \Theta_x} \sum_{i=1}^n \varepsilon_i\langle z_i,\phi\rangle
\end{equation}

\end{lemma}
\begin{proof}
为证结论成立,只需证对$\hat{\theta} \in \Theta$,有$v = (\hat{\theta} - \theta)\|\hat{\theta}-\theta\|_x \in \Theta_x$
\par
显然,$\|v\|_x \leq 1$,并且由$\theta$的定义知,$\hat{\theta}\|\hat{\theta}-\theta\|_x, \theta\|\hat{\theta}-\theta\|_x \in \Theta$,故$v\in \Theta_x$
\end{proof}
\par
将式~\ref{eq:l4}~带入式~\ref{eq:18}~中得到
\begin{equation}\label{eq:20}
\|\hat{\theta} - \theta\|_x \leq 2\cdot \sup_{\phi\in \Theta_x}\mathbb{G}_x(\phi)
\end{equation}
\par
注意到,对$\forall \phi, \phi^\prime \in \Theta_x, \mathbb{G}_x(\phi) - \mathbb{G}_x(\phi^\prime)$是次高斯随机变量,根据Dudley熵积分,有
\begin{equation}\label{eq:21}
\mathbb{E}\sup_{\phi \in \Theta_x} \mathbb{G}_x(\phi) \lesssim \frac{\sigma}{n}\int_0^\infty \sqrt{\log N(\epsilon;\Theta_x,\|\cdot\|_x)} \mathrm{d}\epsilon
\end{equation}
\par
其中,$N(\epsilon;\Theta_x,\|\cdot\|_x)$是$\Theta_x$在$\|\cdot\|_x$中的覆盖数,即满足$\sup_{\phi\in\Theta_x}\inf_{\phi^\prime \in \mathcal{H}}\|\phi-\phi^\prime\|_x \leq \epsilon$的最小集合$\mathcal{H}$的大小。
\par
将式~\ref{eq:20}~带入式~\ref{eq:21}~中得到
\begin{equation}\label{eq:22}
\mathbb{E} [\|\theta-\hat{\theta}\|_x] \lesssim \frac{\sigma}{n}\int_0^\infty \sqrt{\log N(\epsilon;\Theta_x,\|\cdot\|_x)} \mathrm{d}\epsilon
\end{equation}

\par
由于关于$\|\phi\|_x\leq 1$的性质难以推导,这里希望将其转化为$\|\phi\|_2 \leq 1$的性质的推导,由此我们引入限制特征值(restricted eigenvalues)的概念,
\begin{definition}
对于$\{z_i\}_{i=1}^n$,它对参数类$\Phi \subset \mathbb{R}^d$最小和最大限制特征值定义为
\begin{equation}\label{eq:23}
\lambda_{\min}(\{z_i\}_{i=1}^n; \Phi) := \inf_{\phi\in \Phi}\|\phi\|_x^2/\|\phi\|_2^2
\end{equation}
\begin{equation}\label{eq:24}
\lambda_{\max}(\{z_i\}_{i=1}^n; \Phi) := \inf_{\phi\in \Phi}\|\phi\|_x^2/\|\phi\|_2^2
\end{equation}
\end{definition}

\par
对$\rho > 0$,令
\[
\Theta_2(\rho) = \{\phi = \theta-\theta^\prime: \theta, \theta^\prime \in \Theta, \|\phi\|_2\leq \rho\}
\]
\par
可以证明,对于参数类$\Theta(\rho)$,可以给出限制特征值的界
\begin{lemma}
假设$\{z_i\}_{i=1}^n$为次高斯随机向量,对$\forall \rho > 0, \delta \in (*0,1/2], \epsilon\in (0,1/2]$,以下不等式以概率$1-\delta$成立
\begin{equation}
\lambda_{\min}(\{z_i\}_{i=1}^n; \Theta_2(\rho)) \geq \frac{C}{4}-O(Z\sqrt{\log(n/\delta)})\cdot \big(\epsilon + \sqrt{\frac{\log N(epsilon;\Theta_2(1),\|\cdot\|_2\log(1/\delta))}{n}}\big)
\end{equation}
\begin{equation}
\lambda_{\max}(\{z_i\}_{i=1}^n; \Theta_2(\rho)) \leq 4C-O(Z\sqrt{\log(n/\delta)})\cdot \big(\epsilon + \sqrt{\frac{\log N(epsilon;\Theta_2(1),\|\cdot\|_2\log(1/\delta))}{n}}\big)
\end{equation}
\end{lemma}
\par
因为以上得到的对限制特征值和$\|\hat{\theta}-\theta\|_x$的估计都与覆盖数有关,因此若能对覆盖数给出估计,就能得到最终的结果。


\begin{lemma}\label{le:9}
对$\forall \rho > 0, \epsilon \in (0,1]$,有以下成立
\begin{equation}
\log N(\epsilon; \Theta_2(\rho),\|\cdot\|_2) \lesssim m\log(\rho d/s)
\end{equation}
\end{lemma}

\par
通过取$\epsilon = \kappa c/C\sqrt{\log(n/\delta)}$,则可得对充分小的$\kappa > 0$,有$\epsilon \cdot O(C^2 \sqrt{\log(n/\delta)}) \leq c/16$.
\par
当
\[
	n \gtrsim c^{-1}C^2 m\cdot \log(c^{-1}Cd\log(n/\delta))\log^2(n/\delta)
\]
则以概率$1-\delta$,有
\[
	\lambda_{\min}(\{z_i\}_{i=1}^n; \Theta_2(\rho)) \geq c/8, \lambda_{\max}(\{z_i\}_{i=1}^n; \Theta_2(\rho)) \leq 8C
\]
\par
则有$c/16\|\phi\|_2^2 \leq \|\phi\|_x^2 \leq 16C\|\phi\|_2^2, \forall \phi \in \Phi$,进而有
\[
	N(\epsilon;\Theta_x,\|\cdot\|_x) \leq \log N(\epsilon/4\sqrt{C};\Theta_2(4/\sqrt{c},\|\cdot\|_2))
\]
\par
根据引理~\ref{le:9}~有
\[
\log N(\epsilon;\Theta_x,\|\cdot\|_x) \lesssim m \log(c^{-1}Cd/\epsilon)
\]
\par
将其带入式~\ref{eq:22}~并注意到
\[
	\int_0^\infty \sqrt{\max\{\log(1/\epsilon), 0\}}\mathrm{d}\epsilon = O(1)
\]
\par
则可知定理~\ref{theo:1}~成立。
\end{proof}



\section{定理~\ref{theo:2}~的证明}\label{pr:2}

\begin{proof}
为了证明下界,这里先引入以下引理
\begin{lemma}[\citet{tsybakov2008introduction}]
设$\Theta_M = (\theta_0, \theta_1, \cdots, \theta_M)$是一个有限参数集,设$P_j$为$\theta_j, j\in \{0,1,\cdots, M\}$对应的概率分布测度。设$\mathfrak{d}:\Theta_M\times \Theta_M\rightarrow \mathbb{R}^+$定义了$\Theta_M$与$\Theta_M$乘积空间上的一个半距离,并且具有以下的性质:
\begin{enumerate}
\item $\mathfrak{d}(\theta_j,\theta_k)\geq 2\rho > 0, \forall j,k\in\{0,1,\cdots,M\}$;
\item $P_j \ll P_0, \forall j \in \{1,\cdots, M\}$;
\item $\frac{1}{M}\sum_{j=1}^M KL(P_j\|P_0)\leq \gamma\log M$。 
\end{enumerate}
\par
并且以下成立,
\begin{equation}
\inf_{\hat{\theta}} \sup_{\theta_j\ in \Theta_M} \mathrm{Pr}_j[\mathfrak{d}(\hat{\theta},\theta_j)\geq \rho] \geq \frac{\sqrt{M}}{1+\sqrt{M}}(1-2\gamma-2\sqrt{\frac{\gamma}{\log M}})
\end{equation}
\end{lemma}
\par
其中$KL(\cdot,\cdot)$是两个概率分布之间的Kullback-Leibler信息量。
\par
由于在定理~\ref{theo:2}~中假设$x_i$独立同分布于正态分布,噪声也服从正态分布,因此有以下的推论:
\begin{corollary}\label{coro:13}
设$\Theta$为神经网络中的参数集合,记$\rho_{\min} = \min_{j>0}\|\theta_0- \theta_j\|_2/2$,和$\rho_{avg}^2 = \frac{1}{M}\sum_{j=1}^M \|\theta_j-\theta_0\|^2_2$,则对于任意的$n$,有
\[
\inf_{\hat{\theta}} \sup_{\theta_j\in \Theta_M} \mathbb{E}_\mu [\|\hat{\theta}_n-\theta\|_2] \geq \rho_{\min} \times\frac{\sqrt{M}}{1+\sqrt{M}}\bigg(1-\frac{n\rho_{avg}^2}{\sigma^2\log M}-2\sqrt{\frac{n\rho_{avg}^2}{2\sigma^2\log^2 M}}\bigg)
\]
\end{corollary}

\begin{lemma}\label{le:14}
设$\Theta \in \mathbb{R}^d$为神经网络参数集,$\mathcal{I}\subset [d]=\{1,2,\cdots,d\}$,设对任意的$u \in \mathbb{R}^{|\mathcal{I}|}$存在$\theta\in \Theta$使得$u$为$\theta$在$\mathcal{I}$维度上的限制,则存在推论中的有限子集$\Theta^\prime \subset \Theta$,使得$\log M \asymp |\mathcal{I}|, \rho_{\min} \asymp \rho_{avg} \asymp \sqrt{|\mathcal{I}|}\epsilon, \forall \epsilon > 0$.
\end{lemma}
\par
下面的引理说明$\theta$的前$m$个组成成分为自由变量。
\begin{lemma}\label{le:15}
设$\mathcal{I} = \{1,\cdots,m\}$,则对任意的$u\in \mathbb{R}^m$,存在$\theta \in \Theta$满足$\theta|_{\mathcal{I}} = u$
\end{lemma}

\par
令$\epsilon \asymp \sigma \sqrt{m/n}$,则由引理~\ref{le:14}~和~\ref{le:15}~以及推论~\ref{coro:13}~知,定理~\ref{theo:2}~成立。

\end{proof}


